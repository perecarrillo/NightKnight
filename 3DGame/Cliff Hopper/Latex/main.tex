\documentclass{article}
\usepackage{graphicx} % Required for inserting images
\usepackage[catalan]{babel}
\usepackage[letterpaper,top=2cm,bottom=2cm,left=3cm,right=3cm,marginparwidth=1.75cm]{geometry}


\usepackage{enumitem}
\usepackage{amsmath}
\usepackage{graphicx}
\usepackage[colorlinks=true, allcolors=blue]{hyperref}
\usepackage{minted}

\setlength{\parskip}{10pt}
\setlength{\parindent}{0pt}

\title{\Huge\textbf{Cliff Hopper}}
\author{\LARGE Pere Carrillo\\ \LARGE Marc Ordoñez}
\date{4 de juny de 2023}



\begin{document}

\maketitle

\begin{center}
   \Large Facultat d'Informàtica de Barcelona (UPC)
\end{center}

\vspace{5cm}
\begin{figure} [H]
    \centering
    % \includegraphics[width = 0.3\textwidth]{Fib_logo.png} \\
    % \includegraphics[width = 7cm]{upc_logo.png}
    
\end{figure}

\newpage

\section{Introducció}
Cliff Hopper és un videojoc de plataformes 3D per a iPhone que es va llançar el 15 de març de 2017. 
El joc està desenvolupat i produït per Mana Cube, un estudi francès fundat el 2014 que s'especialitza en jocs mòbils. 
Es tracta d'un equip relativament petit. El joc utilitza el motor gràfic Unity i té un estil pixel art retro. 
El joc és controlar un aventurer que fuig d'una roca gegant mentre salta i gira per un camí ple d'obstacles i paranys. 
El joc és gratuït amb publicitat i micropagaments. Una altra informació rellevant és que el joc té diversos personatges i 
escenaris desbloquejables, com ara un ninja, un zombi, una mòmia o un robot. El joc té una puntuació de 4,5 sobre 5 a l'App 
Store i més d'un milió de descàrregues.


\end{document}
